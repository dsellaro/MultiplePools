%==============================================================================
\section{Related Works}
\label{sec:Trabalhos}
%==============================================================================
Particle swarm optimization has become popular due to its simplicity and its effectiveness in wide range of application with low computational cost. The PSO has also been applied to solve NP-Hard problems~\cite{sousa2004} like scheduling~\cite{veeramachaneni2004,Dasgupta2014} and task allocation~\cite{yin2006, zavala2008}.

Many researches about workflow scheduling problem under the cloud computing environment have used PSO, as mentioned below. Pandey et al.~\cite{pandey2010} seeks to minimize the cost of running a single workflow while balancing the task load on available resources. Wu et al.~\cite{wu2010} is concerned with minimizing cost and time, but assumes that a limited set of resources, without taking into account the elasticity provided by cloud computing. Byun et al.~\cite{byun2011} estimate the optimal number of resources that need to be allocated so that the cost of running a workflow is minimized. Its approach takes advantage of the elasticity of cloud resources, but does not consider the heterogeneous nature of computing resources. Jianfang et al.~\cite{Marcon2013} concern with security requirements in order to avoid the risk  of sensitive  data being  leaked or tampered in the process of transmission or execution. Rodrigues and Buyya~\cite{rodriguez2014} approach cost minimization with the deadline constrained. Yassa et al. aims to minimize energy consumption while preserving the users Quality of Service (QoS) preferences, by using an iterative method called Multi-objective Discrete Particle Swarm Optimization (MODPSO) combined with the Dynamic Voltage and Frequency Scaling (DVFS) technique. 

Particle swarm optimization also has been applied for scheduling tasks in grid environments~\cite{zhang2006, liu2010}. Aron et al.~\cite{aron2015} propose the PSO-based hyper-heuristic method that minimizes the time and cost along with optimized utilization of the resources in the grid environment. Sidhu et al. propose a load rebalance algorithm using PSO together with the smallest position value (SPV) technique for task schedule problem~\cite{sidhu2013}. Ramemezani et al.~\cite{ramezani2014} proposed a Task-based System Load Balancing (TBSLB) that achieves system load balancing by migrating tasks from an overloaded VM to another homogeneous VM, instead of migrating the overloaded VM in its entirety.
