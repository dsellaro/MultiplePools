%==============================================================================
\section{Trabalhos Relacionados}
\label{sec:Trabalhos}
%==============================================================================
O agendamento de fluxo de trabalho tem sido amplamente estudado ao longo dos anos, nos quais os algoritmos se concentram na geração de soluções aproximadas ou quase ótimas, por se tratar de um problema não polinomial difícil~\cite{sousa2004}. Pandey et al.~\cite{pandey2010} propõem um algoritmo baseado em PSO para minimizar o custo de execução de um único fluxo de trabalho enquanto equilibra a carga da tarefa nos recursos disponíveis. Wu et al.~\cite{wu2010} usam PSO para produzir um agendamento quase ideal, se preocupando em minimizar custo e tempo, mas assume que um conjunto limitado de recursos, sem levar em conta a elasticidade proporcionada com a computação em das nuvens. O algoritmo de Byun et al.~\citealp{byun2011} estima o número ótimo de recursos que precisam ser alocados para que o custo de execução de um fluxo de trabalho seja minimizado. Sua abordagem aproveita a elasticidade dos recursos da nuvem, mas não considera a natureza heterogênea dos recursos computacionais.