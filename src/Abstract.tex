\begin{abstract}
%-- Context	
 As empresas buscam alternativas tecnológicas que proporcionem competitividade para seus processos de negócios. Dentre essas alternativas, existem as plataformas de integração que conectam as aplicações que compõem seu ecossistema de software, o qual é composto pelas aplicações locais e pelos serviços da computação em nuvem, como SaaS e PaaS, e ainda interage com as mídias sociais.  
 As plataformas de integração implementam soluções de integração, que fazem com que as aplicações funcionem de forma sincronizada, oferendo acesso às informações e funcionalidades de forma rápida e segura. As plataformas de integração geralmente fornecem uma linguagem de domínio específico, um kit de ferramentas de desenvolvimento, um motor de execução e uma ferramenta de monitoramento
 Devido à heterogeneidade do ecossistema de software, há uma constante preocupação com a melhoria das plataformas de integração, a fim de que elas proporcionem a combinação de menor tempo de execução das tarefas de integração com a economia do consumo de recursos computacionais.
 %-- What is the problem?  
A eficiência do motor em agendar e executar tarefas de integração tem um impacto direto no desempenho de uma solução e este é um dos desafios enfrentados pelas plataformas de integração.
 %-- Why it is a problem?
 Nossa revisão da literatura identificou que os motores de integração adotam algoritmos de agendamento de tarefas baseados em políticas, como a de prioridade e a \textit{First-In-First-Out} (FIFO), as quais não se adequam a alguns tipos de cenários, como no caso do aumento da carga de trabalho inicial. Portanto, é oportuno buscar um algoritmo de agendamento de tarefas que otimize desempenho da solução de integração em diferentes cenários.
 %-- Our solution
Esse artigo propõe um algoritmo agendamento de tarefas baseado na técnica de otimização meta-heurística \textit{Particle Swarm Optimization} (PSO), que atribui as tarefas para os recursos computacionais, considerando o tempo de espera na fila de tarefas prontas e a complexidade computacional de cada tarefa, a fim de otimizar o desempenho da solução de integração. 

\textbf{Palavras-chave:} Integração de Aplicações Empresariais; Otimização; Computação em nuvem; Motor de execução; Algoritmo de agendamento de tarefas.
\end{abstract} 